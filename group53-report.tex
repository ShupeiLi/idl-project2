\documentclass{article}

\usepackage{ocencfd}
\usepackage{geometry}
\usepackage{graphicx}
\usepackage{float}
\usepackage{amsfonts}
\usepackage{mathrsfs}
\usepackage{mwe}
\usepackage{subfig}

\title{Introduction to Deep Learning\\Assignment 2} % Sets article title
\author{\textbf{Group 53}\\Chenyu Shi (s3500063); Shupei Li (s3430863); Shuang Fan (s3505847)} % Sets authors name
\documentID{Introduction to Deep Learning Assignment 2} %Should be alphanumeric identifier
\fileInclude{} %Names of file to attach
\date{\today} % Sets date for publication as date compiled

\geometry{a4paper, left=2.5cm, right=2.5cm, top=2.5cm, bottom=2.5cm}

% The preamble ends with the command \begin{document}
\begin{document} % All begin commands must be paired with an end command somewhere

\maketitle % creates title using information in preamble (title, author, date)

\section*{Task 1}

\section*{Task 2}

\section*{Task 3}
\setcounter{section}{3}
\subsection{Datasets}
We use two datasets in Task 3. Firstly, we explore the performance of different model architectures and the effects of different parameters with MNIST data. After that, we leverage the power of generative models on Butterfly \& Moths data. \par
We directly call Tensorflow API to download the MNIST dataset. However, the original dataset is also available on \url{https://deepai.org/dataset/mnist}. MNIST dataset contains 70,000 grayscale images (28 $\times$ 28 $\times$ 1), whose content is handwritten numbers. \par
Butterfly \& Moths is an open source dataset on Kaggle. There are 13,639 RGB images (224 $\times$ 224 $\times$ 3) composed of 100 butterfly or moth species. Link of the dataset is \url{https://www.kaggle.com/datasets/gpiosenka/butterfly-images40-species?resource=download}.

\subsection{Experimental Set-up}
All experiments are deployed on two servers. Server 1
\subsubsection{MNIST}
We modify the 

\subsection{Results}

\subsection{Discussion: Model Comparison}

\section*{Contributions}
\end{document}
